% The \documentclass command is the first command in a LaTeX file.
\documentclass{report}

\pagenumbering{arabic}

\usepackage{graphicx}
%\usepackage{algorithm}
%\usepackage[noend]{algpseudocode}
\usepackage{enumitem}
%\usepackage{epstopdf}
%\usepackage{tikz}
\usepackage{amssymb}
%\usepackage{listings}

\begin{document}

\subsection*{Eddie Davis}
\textbf{CS 573}\\
\textbf{11/6/2017}\\

\section*{HW3: SOLID and Refactoring}

\begin{enumerate}
	\item \textbf{Covariant Argument Type}
	
	This code violates the Liskov Substitution Principle (LSP) because an
	instance of the base class \texttt{CatShelter} cannot reliably be
	substituted for an instance of \texttt{AnimalShelter}. Assume,
	for example, that some client code contained a reference to an \texttt{AnimalShelter} object, and that it was replaced with an instance of
	\texttt{CatShelter}. If the client attempted to invoke the \texttt{putAnimal}
	method with an instance of another \texttt{Animal} subclass, e.g.,
    \texttt{Dog}, this would produce a runtime error as the \texttt{CatShelter}
    class has overriden the \texttt{putAnimal} method with argument type
    \texttt{Cat}, and no method signature \texttt{putAnimal(Dog dog)} exists.
	
	In subcontracting terms, this is a problem because the \texttt{Cat} type
	in the \texttt{putAnimal} method imposes a stricter precondition,
	but overriden methods in subclasses should have weaker preconditions,
	otherwise the subcontract between the client and supplier is violated.
	
	The \texttt{Cat} return type of the overriden \texttt{getAnimal()} method
	is not a violation of the LSP because if the client with its
	\texttt{AnimalShelter} object invokes that method and assigns it 
	to an object of type \texttt{Animal}, a \texttt{Cat} object is-a \texttt{Animal} and should behave like one.
	
	\item \textbf{Open-Closed Principle}
	
	The \texttt{AreaCalculator.Area(Shape[] shapes)} method violates the 
	Open-Closed Principle (OCP) because the code within the \texttt{for}
	loop explicitly references the specific subclasses of the \texttt{Shape}
	class, \texttt{Rectangle} and \texttt{Circle}. This means that in order
	to add another type of shape, e.g., \texttt{Triangle}, requires the
	\texttt{Area} method to be modified. Thus, the class is \textit{not}
	closed for modification.
	
	The code could be improved by adding an \texttt{Area} method to the
	\texttt{Shape} abstract class, which is then implemented by each of the
	subclasses. Then the \texttt{AreaCalculator.Area} method only needs to
	call the \texttt{Area} method of each \texttt{Shape} object. Please see
	the following code for example.
	
	\begin{verbatim}
public abstract class Shape {
    public double Area();
}
    
public class Rectangle extends Shape {
    public double Area() {
        return rectangle.Width * rectangle.Height;
    }
}
    
public class Circle extends Shape {
    public double Area() {
        return circle.Radius * circle.Radius * Math.PI;
    }
}
    
public class Triangle extends Shape {
    public double Area() {
        return 0.5 * rectangle.Width * rectangle.Height;
    }
}

public class AreaCalculator {
    public double Area(Shape[] shapes) {
        double area = 0.0;
        for (Shape shape: shapes) {
            area += shape.Area();
        }
        return area;
     }
}
	\end{verbatim}

	\item \textbf{Refactoring}

    The TicTacToe program has been refactored in the following ways:
    
    \begin{enumerate}
\item The \texttt{TTTGraphics2P} constructor is too long (more than a page). It has been 
refactored by adding the \texttt{addMouseListener} method.
 
\item The \texttt{DrawCanvas} class is poorly named (it is a verb), renamed to
\texttt{DrawingCanvas}.
 
\item The \texttt{hasWon} method is poorly named, it has been renamed to \texttt{isWon}
to match nomenclature of the \texttt{isDraw} method.
 
\item The \texttt{DrawingCanvas.paintComponent} method is too long and compartmentalized,
its responsibilities have split into three methods: \texttt{drawGridLines, drawSeeds}, and
\texttt{updateStatusBar} methods.
 
\item The code in the \texttt{drawSeeds} and \texttt{updateStatusBar} methods rely on
\textbf{if} statements to determine behavior based on the type of an enumeration (e.g., \texttt{Seed}, \texttt{GameState}). This means that any time a new type needs to be
added (such as a three player version of TTT with \texttt{Star} shape added), all of the supporting code would need to be modified. This would result in a violation of the Open-
Closed Principle. Instead, the \texttt{Seed} and \texttt{GameState} should be converted
into class hierarchies, allowing the behavior differentiation to be handled via
polymorphism, rather than control statements. The code has been refactored, adding
classes \texttt{Seed, Empty, Cross, Nought}, and \texttt{GameState}, 
 
\item The \texttt{isWon} method uses a complex boolean expression to determine whether
the game
 is won. The expression should be decomposed into a set of boolean methods.
 The row and column values are also hard-coded, meaning the code can only support a 3X3
 grid size. The expression has been decomposed into four methods: \texttt{isRowWon,
 	isColWon, isFwdDiagWon}, and \texttt{isRevDiagWon}. Each contains a loop
 dependent on the number of rows and columns, so would work for a larger grid size as well.
    \end{enumerate}

Please find the refactored code in the attached file \texttt{TTTGraphics2P.java}.

\end{enumerate}

\end{document}
