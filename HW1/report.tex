% The \documentclass command is the first command in a LaTeX file.
\documentclass{bsu-ms}
%\documentclass[project]{bsu-ms}  % for project reports

\pagenumbering{arabic}

\usepackage{graphicx}
%\usepackage{algorithm}
%\usepackage[noend]{algpseudocode}
\usepackage{enumitem}
%\usepackage{epstopdf}

\begin{document}

\subsection*{Eddie Davis}
\textbf{CS 573}\\
\textbf{9/25/2017}\\

\section*{HW1: Petri Nets}

\begin{enumerate}
	\item Given the first Petri net,
	\begin{enumerate}[label=(\alph*)]
		\item The formal definition is given by \\
		$PN_1$= ($P$, $T$, $F$, $W$, $M_0$) $\mid$ \\
		$P$ = \{$p_1, p_2, p_3, p_4, p_5, p_6, p_7, p_8$\}, \\
		$T$ = \{$t_1, t_2, t_3, t_4, t_5, t_6$\}, \\
		$F$ = \{ ($t_1$, $p_1$), ($p_3$, $t_1$), ($p_1$, $t_2$), \\
		($p_1$, $t_2$), ($t_2$, $p_2$), ($t_2$, $p_4$),
		($p_2$, $t_3$), ($p_2$, $t_3$), \\
		($p_5$, $t_3$), ($t_3$, $p_3$), ($t_4$, $p_5$),
		($t_4$, $p_8$), ($p_7$, $t_4$), \\ ($p_4$, $t_5$),
		($p_6$, $t_5$), ($t_6$, $p_6$), ($p_8$, $t_6$)
		 \}, \\
   	    $W$ = \{ 1, 1, 1, 1, 1, 1, 1, 1, 1, 1, 1, 1, 1, 1, 1, 1, 1 \}, \\
        $M_0$ = \{ 1, 0, 0, 0, 0, 1, 0, 0 \}
		\item The coverability tree looks like...
		\item The Petri net is/not bounded because...
		\item The Petri net is/not live/L0-live/L1-live because...
	\end{enumerate}
	
	\item Given the second Petri net,
	\begin{enumerate}[label=(\alph*)]
		\item The formal definition is given by \\
		$PN_2$= ($P$, $T$, $F$, $W$, $M_0$) $\mid$ \\
		$P$ = \{$p_1, p_2, p_3, p_4$\}, \\
		$T$ = \{$t_1, t_2, t_3$\}, \\
		$F$ = \{ ($t_1$, $p_1$), ($p_1$, $t_1$), ($p_2$, $t_1$),
		($p_3$, $t_1$), \\ ($t_2$, $p_2$), ($t_2$, $p_3$),
		($p_4$, $t_2$), ($p_3$, $t_3$), ($t_3$, $p_4$) \} \\
		$W$ = \{ 1, 1, 1, 1, 1, 1, 1, 1, 1 \}, \\
        $M_0$ = \{ 1, 0, 1, 0 \}
		\item The coverability tree looks like...
		\item The Petri net is/not bounded because...
		\item The Petri net is/not live/L0-live/L1-live because...
	\end{enumerate}
\end{enumerate}

\end{document}
