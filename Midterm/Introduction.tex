%% INTRODUCTION

\section{Introduction}

The goal of this project is to apply object-oriented design and agile development principles to
develop a graphical environment for the development and simulation of Petri Nets.
A Petri Net (PN) is a mathematical modeling language represented as a directed, bipartite graph. The
graphs can be used to model software processes, state machines, formal language semantics,
parallel and distributed systems, and many other computer science applications. The PN
abstraction provides a higher level than finite state automata, able to model nondeterministic
or stochastic processes.

A PN graph consists of two types of nodes, with \textit{places} represented by circles, and
the \textit{transitions} between them by rectangles. Places and transitions are connected by
directed arrows called \textit{arcs}. An arc connecting a place to a transition is called a
\textit{source} arc, and one connecting a transition to a place is known as a \textit{destination}.
Places can contain a number of \textit{tokens}, represented by an integer value, $t \geq 0$.
An arcs can also have a \textit{weight}, an integer value, $w \geq 1$, that defaults to 1, and
determines the number of tokens that move from one place to another between the transition. A
transition may have any number of input arcs, and is \textit{firable} when all of the connected
places contain the number of tokens specified by the arc weights. When the transition fires, tokens
are moved from the destination place to the source place. A PN configuration, i.e., the number of
tokens in each place is known as a \textit{marking}, and each PN has an initial one. An ordered
set of transition firings is called a \textit{firing sequence}.

The formal definition of a Petri Net, $PN$ with a set of places $P$ of size $m$, set of $T$
transitions of size $n$, set of transition functions $F$ representing the arcs, and weight function
$W$, assigning a weight to each arc, is given below. The initial marking is denoted as $M_0$, and all
subsequent markings as $M_1, M_2, ...$. The firing sequence with $p$ firings is denoted by $\sigma$.

\begin{align*}
PN = (N, M_0) \\
N = (P, T, F, W) \\
P = \{ p_1, p_2, ..., p_m \} \\
T = \{ t_1, t_2, ..., t_n \} \\
F \subseteq (P \times T) \cup (T \times P), F \neq \emptyset \\
W: F \rightarrow \{ 1, 2, 3, ... \} \\
M_0: P \rightarrow \{ 1, 2, 3, ... \} \\
\sigma = (M_0, t_1M_1, t_2M_2, ..., t_pM_p) \\
\end{align*}

%$\newline$

The remainder of this report describes the design and development of the Petri Net simulator.
Section 2 covers the team organization, Section 3 the product backlog, i.e., user stories,
Section 4 details the object oriented design, Section 5 the pair development summary, Section 6
the overall development process, and finally in Section 7, each author provides his lessons
learned thus far.


