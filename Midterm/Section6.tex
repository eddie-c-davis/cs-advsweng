%% LESSONS LEARNED

\section{Lessons learned}

At the beginning of this project, I had only recently become aware of Petri Nets, while researching
related work for a paper on macro dataflow graphs modeling distributed memory applications. In the
era of big data, heterogeneous computing platforms, and the apparent bottleneck of the
memory subsystem in multi-core architectures, modeling systems based on the availability of data
are becoming increasingly important. To that end, I have learned much more about the applications of
Petri nets to complex problems such as \textit{Dining Philosophers} and \textit{Producers-Consumers}.

In development terms, I have learned much more about dividing a large project into multiple components
and the challenges of developing those components in a team environment. During much of my professional
and academic life, much of my work has taken place in isolation, collaborating mostly at a high level,
either having complete control to modify code in an application, or none at all. The need to collaborate
at a low level in a tight knit development group has been enlightening.

The way in which the concepts covered during lecture are able to be applied to the class project
enable a hands-on learning approach that make the course more effective. One specific lesson learned
is the need to be cautious when developing international code. It is important to pay attention
to details like number formats, a comma is used to represent the radix in the EU, instead of a decimal
point as in the US. A file written with the comma format for a floating point number will fail when
read by a system expecting a decimal point instead. Such details highlight the need for comprehensive
unit testing, particularly for human-computer interaction.
