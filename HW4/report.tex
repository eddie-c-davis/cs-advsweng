% The \documentclass command is the first command in a LaTeX file.
\documentclass{report}

\pagenumbering{arabic}

\usepackage{graphicx}
%\usepackage{algorithm}
%\usepackage[noend]{algpseudocode}
\usepackage{enumitem}
%\usepackage{epstopdf}
%\usepackage{tikz}
\usepackage{amssymb}
\usepackage{booktabs}
%\usepackage{listings}

\begin{document}

\textbf{Eddie Davis}
\textbf{CS 573}\\
\textbf{12/4/2017}\\

\section*{HW4: Testing}

\begin{enumerate}
	\item \textbf{Mutation Testing and Fault Detection Conditions}

	Please see the \texttt{TestCoins} and \texttt{TestDrinks} classes in the
	\texttt{VendingMachine.test} package.
	The mutation score for this set of unit tests is 0.804.
	Most of the living mutants are the result of post-increment or decrement operators (++ or --) appended
	to the end of local variables or arguments used in calculations.

    \begin{table*}
    	\centering
    	\caption{Listing of living mutants. \label{tbl:mutants}}
    	\begin{tabular}{|l|l|l|}
    		\hline
    		\textbf{Live mutant \#} & \textbf{Description of live mutant} & \textbf{Why it was not killed} \\
    		AOIS\_20	& calculateChange(int price) \{  change = deposit – price--; …	& price-- is a post decrement applied to an argument so does not affect the output. \\
    		AOIS\_16	& calculateChange(int price) \{ change = deposit-- - price;…	& As above, post-decrement applied so output is not affected. \\
    		AOIS\_19	& calculateChange(int price) \{ change = deposit – price++;… &	This time, a post-increment is applied to price, but output remains unaffected. \\
    		AOIS\_15 & calculateChange(int price) \{  change = deposit++ - price; &	Another case of post-increment applied that does not affect the output of the calculation. \\
    		AOIS\_4	& setDrink(…) \{ … coffee.setCount( newCount-- ); ... \} & Yet another post-decrement of a local variable that does not affect the class state. \\
    		AOIS\_1 & setDrink(…) \{ … coffee.setPrice( newPrice++ ); … \} & Another case of post-incrementing a local variable without affecting the class state. \\
    		AOIS\_3	setDrink(…) \{ … coffee.setCount( newCount++ ); …\} & Another case of post-incrementing a local variable without affecting the class state. \\
    		AOIS\_6 & setDrink(…) \{ …  juice.setPrice( newPrice-- ); … \} & Another case of post-decrementing a local variable without affecting the class state. \\
    		AOIS\_11 & setDrink(…) \{ … soda.setCount( newCount++ ); … \} & Another case of post-incrementing a local variable without affecting the class state. \\
    		AOIS\_12  & setDrink(…) \{ … soda.setCount( newCount-- ); … \} & Another case of post-decrementing a local variable without affecting the class state. \\
    		\hline
    	\end{tabular}
    \end{table*}

	\item \textbf{Automated GUI Testing}
	
	Please see the \texttt{test.TestGUI} class in the \texttt{VendingMachine} package.
	
	\item \textbf{Model-based testing with finite state machines}
	
	Please see the \texttt{test.TestFSM} class in the \texttt{VendingMachine} package.
    The mutation score for this set of unit tests is 0.698.
	
\end{enumerate}

\end{document}
