% The \documentclass command is the first command in a LaTeX file.
\documentclass{report}

\pagenumbering{arabic}

\usepackage{graphicx}
%\usepackage{algorithm}
%\usepackage[noend]{algpseudocode}
\usepackage{enumitem}
%\usepackage{epstopdf}
%\usepackage{tikz}
\usepackage{amssymb}
%\usepackage{listings}

\begin{document}

\subsection*{Eddie Davis}
\textbf{CS 573}\\
\textbf{10/16/2017}\\

\section*{HW2: Design Principles}

\begin{enumerate}
	\item The $QUEUE$ ADT specification is given below with \texttt{new, add, poll, peek}, and \texttt{isEmpty} functions, and
	variables $x$, of generic type $G$, and $q$, type $QUEUE[G]$.
	
	\begin{enumerate}
		\item Type: \textit{QUEUE[G]}, all queues of arbitrary type $G$.
		\item Functions:
		\begin{itemize}
			\item \texttt{new}: $\rightarrow QUEUE[G]$ \\
			$creator$ function
			\item \texttt{add}: $QUEUE[G] \times G \rightarrow QUEUE[G]$ \\
			$command$ function
			\item \texttt{poll}: $QUEUE[G] \rightarrowtail G$ \\
			$command$ function			
			\item \texttt{peek}: $QUEUE[G] \rightarrowtail G$ \\
    		$query$ function
			\item \texttt{isEmpty}: $QUEUE[G] \rightarrow BOOLEAN$ \\
    		$query$ function			
		\end{itemize}
		
		\item Axioms		
		$\forall x:G, q:QUEUE[G]$,
			\begin{enumerate}
				\item \texttt{poll}(\texttt{add}(\texttt{new}, $x$)) = $x$
				\item \texttt{peek}($q, x$) = \texttt{poll}($q, x$)
				\item \texttt{isEmpty}(\texttt{new}) = \texttt{true}
				\item \texttt{not isEmpty}(\texttt{add}($q, x$)) = \texttt{true}
        	\end{enumerate}		

		\item Preconditions		
		\begin{itemize}
			\item \texttt{poll}($q: QUEUE[G]$) \texttt{requires not isEmpty}($q$)
            \item \texttt{peek}($q: QUEUE[G]$) \texttt{requires not isEmpty}($q$)
		\end{itemize}
	\end{enumerate}

	\item 	
	\begin{enumerate}
		\item 
		The \texttt{ATM} class mixes business logic and networking. By implementing the \texttt{Runnable} interface, it is also mixes threading logic. Otherwise, 
		it provides a good abstraction, it is a noun with method names corresponding
		to the proper verbs describing its functionality, e.g., \texttt{switchOn},
		\texttt{switchOff}, \texttt{performStartup}, and \texttt{performShutdown}. \\
		
		The \texttt{Money} class is also a good abstraction, however one does question
		whether it is necessary. Most of its methods involve mathematical
    	(\texttt{add}, \texttt{subtract}) or comparative (\texttt{lessEqual})
    	functions that could be supported by a standard library class such as \texttt{Long}. Only the \texttt{toString} method is money specific,
    	and could be supported by the \texttt{Locale} class. \\
    	
    	Similarly, the \texttt{Card} class could easily be replaced by an
    	integer primitive \texttt{int}. However, if other properties or methods
    	needed to be added, it would be a reasonable abstraction.
		
		\item The variables \texttt{ACCOUNT\_NUMBER}, \texttt{WITHDRAWALS\_TODAY}, \\ \texttt{DAILY\_WINTHDRAWAL\_LIMIT},
		and \texttt{BALANCE}
		in the \texttt{SimulatedBank} could be encapsulated into
		a single class called \texttt{SimulatedAccount} with
		attributes \texttt{number}, \texttt{withdrawalsToday},
		\texttt{dailyWithdrawalLimit}, and \texttt{balance}.
        The corresponding operations, such as \texttt{add} and
        \texttt{subtract} could be replaced with methods like
        \texttt{deposit} and \texttt{withdraw}. This abstraction
        would provide a clear separation of concerns and 
        application of object-oriented design principles.
	\end{enumerate}

	\item The command-query separation principle states that every method should either be a command that performs an action
	(mutator), or a query that returns data (accessor), but not both. The \texttt{ATM} class does \textbf{not} violate the CQS principle.
	All of the \textit{get} methods (e.g., \texttt{getID},
	\texttt{getPlace}, \texttt{getBankName}) return references to the
	corresponding instance variables and do nothing more. While the
    command	methods (e.g., \texttt{switchOn}, \texttt{switchOff},
	\texttt{performStartup}, modify the object state, but \texttt{performShutdown}) return nothing (\texttt{void} type).
		
	\item Consider the precondition of method subtract in class Money:
	public void subtract(Money amountToSubtract)
	\begin{enumerate}
	    \item The \texttt{this.cents - amountToSubtract.cents} $\geq$ 0 precondition of the \texttt{subtract} method
	    is a demanding precondition because the responsibility
	    to satisfy it is deferred to the client. If the client
	    does not satisfy it, the \texttt{cents} variable will
	    be negative which may produce incorrect behavior
	    depending on the application.\\
	    The precondition is reasonable and available because
	    the client need only call the \texttt{Money.lessEqual}
	    method before invoking \texttt{subtract} to determine
	    whether the precondition is satisifed.
	    
	    \item The \texttt{CashDispenser.dispenseCash} client
	    method does not check the \\ \texttt{subtract} precondition
	    but instead defers it to its own precondition that
	    \texttt{amount} $\leq$ \texttt{cashOnHand}, as does
	    the \texttt{Simulation.dispenseCash} method. The
	    \texttt{SimulatedBank.withdrawal} method checks the
	    precondition and returns a \texttt{Failure} state if
	    it is violated, as does the \texttt{transfer} method.
	    The \texttt{Withdrawal.completeTransaction} method calls \\
	    \texttt{CashDispenser.dispenseCash} without checking
	    the precondition, so the possibility of a violation
	    exists.
	\end{enumerate}
	
	\item 
	\begin{enumerate}
		\item Yes, it is possible for the \texttt{switchOn} variable to be
		\texttt{false} and \texttt{state} equal to
		\texttt{SERVING\_CUSTOMER\_STATE} in the \texttt{ATM} class. \\
		Consider the code snippet below:
				
		\begin{verbatim}		
		ATM atm = new ATM(1, "SUB", "MyBank", "127.0.0.1");    // 1
        // switchOn==false AND state==OFF_STATE
		atm.switchOn();	                                       // 2
		// switchOn==true AND state==OFF_STATE	     
		atm.run();                                             // 3
		// switchOn==true AND state==IDLE_STATE
		atm.cardInserted();                                    // 4
		// switchOn==true AND state==IDLE_STATE
		atm.run();                                             // 5
		// switchOn==true AND atm.state==SERVING_CUSTOMER_STATE
		atm.switchOff();                                       // 6
		// switchOn==false AND state==SERVING_CUSTOMER_STATE
		atm.run();                                             // 7
		// switchOn==false AND state==IDLE_STATE
		...
		\end{verbatim}		
		
		The comment to the right of each statement indicates the line number,
		and the comment after each statement the values of the
		\texttt{switchOn} and \texttt{state} variables after each line is
		executed. There is a time interval after \texttt{cardInserted} and
		\texttt{run} have been called (lines 4 and 5), that if the
		\texttt{switchOff} method is called (line 6), that \texttt{switchOn}
		will be \texttt{false}, and \texttt{state} will be
		\texttt{SERVING\_CUSTOMER\_STATE} before \texttt{run} is called
		again (line 7) and sets \texttt{state} back to \texttt{IDLE\_STATE}.
		
	    \item Yes, there are four methods that could possiblet set
	    \texttt{this.cents} less than zero in the \texttt{Money} class. 
	    The constructor \texttt{Money(int dollars, int cents)}, does not ensure
	    that either \texttt{dollars} or \texttt{cents} are greater than or equal
	    to zero. Nor does the copy constructor \texttt{Money(Money toCopy)}
	    check that \texttt{toCopy.cents} is not less than zero. The \texttt{add}
	    method does not ensure that \texttt{amountToAdd} is non-negative. Nor
	    does the \texttt{subtract} method ensure that \texttt{amountToSubtract}
	    is not greater than \texttt{this.cents}, despite having a precondition
	    comment stating that this must be so. A class invariant would be
	    needed to solve these problems.
    \end{enumerate} 
    	
    \item In general, the class hierarchy of \texttt{Transaction} as an
    abstract class, and \\ \texttt{Withdrawal}, \texttt{Transfer},
    \texttt{Deposit}, and \texttt{Inquiry} as its concrete subclasses is a
    good application of inheritance, because they all satisfy an ``is-a''
    relationship. The \texttt{getSpecificsFromCustomer} and
    \texttt{completeTransaction} methods are abstract and is implemented
    by each of the subclasses. The rule of change is satisfied as every
    \texttt{Withdrawal} is always a \texttt{Transaction} for example, and
    the relationships should not change at runtime. The polymorphism rule
    is also satisfied because a variable of type \texttt{Transaction} can
    become a \texttt{Withdrawal}, \texttt{Transfer}, \texttt{Deposit}, or
    \texttt{Inquiry} at runtime (dynamic binding).
    
    However, there is a violation of good abstraction principles in the
    static \\ \texttt{makeTransaction} method of the \texttt{Transaction}
    class, is it explicitly refers to its subclasses. This is not a good use of
    inheritance, as a superclass should have no knowledge of its subclasses.

\end{enumerate}

\end{document}
